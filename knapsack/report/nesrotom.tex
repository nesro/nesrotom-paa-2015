\documentclass[12pt,a4paper]{article}

\usepackage[utf8]{inputenc}
\usepackage{graphicx}
\usepackage{float}

\begin{document}

\title{MI-PAA 2015 1.ukol}
\author{Tomas Nesrovnal\\nesrotom@fit.cvut.cz}
\date{\today}
\maketitle

\section{Specifikace ulohy}
Problem batohu.

\section{Rozbor moznych variant reseni}
Ulohu muzu resit hrubou silou. Ziskam tam presny vysledek, ale vypocet bude
pomaly. Dalsim resenim je pouzit heuristiku, jejiz vyledek nebude nejlepsi mozny
ale vypocet probehne rychle.

\section{Ramcovy popis postupu reseni}
\subsection{Hruba sila}
Zkusim vsechny moznosti a vyberu tu nejlepsi.
\subsection{Heuristika}
Vkladam do bahothu nejlepsi predmety s pomerem cena/vaha, dokud
mi jeste staci kapacita.

\section{Popis kostry algoritmu}
\subsection{Hruba sila}
Vytvorim pole, ktere udava ktery predmet je v batohu. Rekurzivne zkousim
vsechny moznosti (zavolam rekurzi bez prvku, pak prvek pridam a zavolam rekurzi znovu).
Ulozim si nejlepsi reseni.

Pokud ve stromu reseni narazim na to, ze se do batohu uz vic nevejde, vetev zariznu.
\subsection{Heuristika}
Seradim si pole s predmety podle pomeru cena/vaha. Cele pole sestupne prochazim a pokud se tam predmet vejde, tak ho tam vlozim.

\section{Namerene vysledky}

\subsection{Spravnost vysledku}
Pomoci skriptu byla overena spravnost vysledku (porovnanim s referencnim resenim).

\subsection{Na cem bylo mereno}
Intel(R) Core(TM) i3-2328M Processor (3M Cache, 2.20 GHz), gcc 4.9.2 (-Ofast), OS GNU/Linux Lubuntu 14.04 64bit

\subsection{Grafy}
Grafy byly vygenerovany skripty (tests/run.sh a time.sh). Cas byl meren pomoci knihovny OpenMPI.

\begin{figure}[H]
	\caption{Doba behu reseni heuristikou. 50k opakovani.}
	\includegraphics{./time_h.png}
\end{figure}

\begin{figure}[H]
	\caption{Doba behu reseni hrubou silou. Pouze jedno opakovani, presto instance o 32 dvou prvcich trvala 22 minut. }
	\includegraphics{./time_b.png}
\end{figure}

\begin{figure}[H]
	\caption{Chyba heuristiky. Z kazdeho batohu spocitana relativni chyba. V grafu jsou pak secteny relativni chyby pro celou sadu.}
	\includegraphics{./err_h.png}
\end{figure}

\section{Zaver}
Vysledky se shoduji s rozborem reseni. Hruba sila je opravdu pomala a byt jednoducha heuristika nevraci zas tak spatne reseni.

\end{document}
